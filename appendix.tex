\section{Miscellaneous results}
In this section, we prove results that we need in the main text, but don't fit in the flow of the text. These are well-known results, which nevertheless aren't usually covered in introductionary algebra courses. Hence, we present them here.

\subsection{The generalized division algorithm}\label{app:pseudo}
Ususally, the theory of Gröbner bases is developed for polynomial rings over a field. However, lots of the usual results hold in polynomial rings over a general commutative, Noetherian ring $A$ as well. Here, we give some of the results we need.

\begin{theorem}
  Let $f \in A[X]$ and $G = \{g_{1}, \dots, g_{n}\} \subset A[X]$. Then there exists $\{h_{1}, \dots, h_{n}\} \subset A[X]$ and $r \in A[X]$ such that
  \[f = r + \sum_{i=1}^{n} h_{i} g_{i}\]
  and the following properties are satisfied:
  \begin{enumerate}
    \item $\LM(h_{i} g_{i}) \leq \LM(f)$ for all $i \in \{1, \dots, n\}$.
    \item No term of $r$ is divisible by $\LT(g_{i})$ for any $i$,
    \item $\coef(h_{i} g_{i}, m) \in \langle \coef(f, m') \mid m' \geq m \rangle$ for all $i \in \{1, \dots, n\}$ and all monomials $m$.
  \end{enumerate}
\end{theorem}
\begin{proof}
  First, we present the division algorithm to compute such a representation. To start, let $f^{0} = f$, $r^{0} = 0$ and $h_{1}^{0} = h_{2}^{0} = \dots = h_{n}^{0} = 0$. Then we iteratively define the state at step $i$ in terms of the state at step $i-1$:
  \begin{itemize}
    \item If $f^{i-1} = 0$, we are done. Set $r = r^{i-1}$ and set $h_{j} = h_{j}^{i-1}$ for all $j \in \{1, \dots, n\}$.
    \item If there is some $g_{j} \in G$ such that $\LT(g_{j}) \mid \LT(f^{i-1})$, then find a $\gamma$ such that $\LT(g_{j})\gamma = \LT(f^{i-1})$ and set $h_{j}^{i} = h_{j}^{i-1} + \gamma$, set $f^{i} = f^{i-1} - \gamma g_{j}$, set $r^{i} = r^{i-1}$, and set $h_{l}^{i} = h_{l}^{i-1}$ for $l \neq j$.
    \item If no $g \in G$ satisfies $\LT(g) \mid \LT(f^{i-1})$, then set $r^{i} = r^{i-1} + \LT(f)$, set $f^{i} = f^{i-1} - \LT(f^{i-1})$ and set $h_{j}^{i} = h_{j}^{i-1}$ for $j \in \{1, \dots, n\}$.
  \end{itemize}

  Since for all $i$, we have $\LM(f^{i}) < \LM(f^{i-1})$ and $<$ is a well-order, this procedure must terminate eventually. The equality
  \[f = r + \sum_{j=1}^{n} h_{j} g_{j}\] follows from the fact that
  \[f - f^{i} = r^{i} + \sum_{j=1}^{n} h^{i}_{j} g_{j}\]
  at every step $i$ of the algorithm, and when the algorithm terminates $f^{i} = 0$.

  The two first properties of the division are invariants for the algoritm. Since we have $\LM(f^{i}) \leq \LM(f)$ for all $i$, property (1) follows from the construction of the $h_{j}$'s. Property (2) is an invariant of $r^{i}$.

  The final property follows from the invariant, that at every step $i$, we have that $\coef(f^{i}, m) \in \langle \coef(f, m') \mid m' \geq m \rangle$. Indeed, it is true at step $i=0$. At step $i$, note that when $\LT(g_{j}) \gamma = \LT(f^{i})$, then $\coef(g_{j} \gamma, m) \in \langle \LC(f^{i}) \rangle$. Since $\LM(g_{j} \gamma) \geq m$ for every monomial $m$, that occurs in a term of $f^{i}$, we get that $\coef(f^{i} - g_{j} \gamma) \in \langle \coef(f^{i}, m') \mid m' \geq m \rangle$.
\end{proof}




\subsection{The nilradical}
The nilradical is the ideal of all nilpotent elements of a ring. It is widely used in the study of general rings. In our case, where the base ring is assumed to have no nilpotents, it is zero, but we still need a different characterization of it.

\begin{definition}[Nilradical]
  Let $A$ be a commutative ring. Then the ideal \[\sqrt{\langle 0 \rangle} = \{a \in A \mid \exists n \in \N : a^{n} = 0\}\] is called the \textit{nilradical}.
\end{definition}

\begin{theorem}\label{thm:nil_rad_is_cap_primes}
  Let $A$ be a commutative ring, and let $\Spec(A)$ be the set of prime ideals of $A$. Then
  \[\sqrt{\langle 0 \rangle} = \bigcap_{\p \in \Spec(A)} \p\]
\end{theorem}
\begin{proof}
  First, a quick induction proof gives that every nilpotent element is in every $\p \in \Spec(A)$. Indeed, $0 \in \p$ and if $a^{n} = 0 \in \p$, then either $a$ or $a^{n-1}$ is in $p$, since $\p$ is prime. By induction, $a \in \p$.

  For the converse inclusion,
\end{proof}




\subsection{Homogenous ideals}

Here, we present a basic lemma about homogenous ideals.

\begin{lemma}\label{lem:homo_components}
  Let $I \subset A[X]$ be a homogenous ideal and let $f \in I$. Writing
  \[f = \sum_{i} f_{i}\]
  where each $f_{i}$ is homogenous, each $f_{i} \in I$.
\end{lemma}
\begin{proof}
  Let $\{g_{1}, \dots, g_{n}\} \subset I$ be a finite set of homogenous generators of $I$, and let $f \in I$. Then we can write
  \[f = \sum_{i=0}^{n} h_{i} g_{i}\]
  for some $h_{i} \in A[X]$. Consider a single term of this sum, which we can write as
  \[h_{i} g_{i} = \sum_{j} a_{i,j}X^{v_{i,j}} g_{i}, \quad \text{where } h_{i} = \sum_{j} a_{i.j}X^{v_{i,j}}.\]
  Each term of this sum is homogenous and $a_{i,j} X^{v_{i,j}} g_{i} \in I$. Since
  \[f = \sum_{i,\,j} a_{i,j} X^{v_{i, j}} g_{i}\]
  is a sum of homogenous polynomials, and each term of the sum is homogenous and in $I$, each homogenous component of $f$ is in $I$.
\end{proof}
