\section{Preliminaries}
This project will assume familiarity with commutative ring theory and multivariate polynomials over fields. A familiarity with Gröbner bases will be beneficial, but we will introduce the necessary notations and definitions. For a comprehensive reference on Gröbner bases, see~\cite{IVA}.

Let $A$ be a Noetherian, commutative ring and $X = (x_{1}, x_{2}, \dots, x_{n})$ be an ordered collection of variables. We denote the ring of polynomials in these variables $A[X]$. Given two disjoint sets of variables $X$ and $Y$, we will use $A[X, Y]$ to mean $A[X \cup Y]$, which is isomorphic to $A[X][Y]$. A \textit{monomial} is a product of variables and a \textit{term} is a monomial times a coefficient. We denote a monomial as $X^{v}$ for some $v \in \N^{n}$. For a polynomial \[f = \sum_{v \in \N^{n}} a_{v}X^{v}\] we denote the coefficient of the term $t = a_{v}X^{v}$ by $\coef(f, X^{v}) = a_{v}$. Note, that $\coef(f, X^{v})$ is defined for any monomial $X^{v}$, but we can only have $\coef(f, X^{v}) \neq 0$ for finitely many $v \in \N^{n}$. For example in $f = xy + 2$, we have $\coef(f, xy) = 1$, $\coef(f, 1) = 2$ and $\coef(f, x) = 0$.

\begin{definition}[Monomial order, leading term]
  A \textit{monomial order} is a well-order\footnote{A total order, for which any chain $a > b > c > \dots$ must be finite.} $<$ on the set of monomials satisfying that $u < v \implies wu < wv$ for all monomials $u, v, w$.

  Given a monomial order $<$ and a polynomial $f \in A[X]$, the \textit{leading term} of $f$ is the term with the largest monomial w.r.t. $<$ and is denoted by $\LT_{<}(f)$. If $\LT_{<}(f) = a\cdot m$ for some monomial $m$ and $a \in A$, then we denote $\LM_{<}(f) = m$ and $\LC_{<}(f) = a$. If $<$ is clear from context, it will be omitted.
\end{definition}

These definitions naturally extend to sets of polynomials, so given a set of polynomials $F \subset A[X]$, we denote $\LM_{<}(F) := \{\LM_{<}(f) \mid f \in F\}$. An example of a monomial order is the lexicographic order, where $X^{v_{1}} > X^{v_{2}}$ if and only if the first non-zero entry of $v1 - v2$ is positive. With this, we can give the definition of a Gröbner basis. Usually, this is done over a field. For a reference on Gröbner bases over rings, see~\cite{loustaunau1994introduction}. The standard reference on Gröbner bases is~\cite{IVA}.

\begin{definition}[Gröbner basis]
  Let $G \subset A[X]$ be a finite set of polynomials and $<$ be a monomial order. We say $G$ is a \textit{Gröbner basis} if the ideal generated by leading terms of $G$ is equal to the ideal generated by leading terms of the ideal generated by $G$, i.e.
  $\langle \LT_{<}(G) \rangle = \langle \LT_{<}(\langle G \rangle) \rangle$.
\end{definition}

Note, that if $A$ is a field, then it is enough that $\langle \LM_{<}(G) \rangle = \langle \LM_{<}(\langle G \rangle) \rangle$. We say $G$ is a Gröbner basis for an ideal $I$ if $G$ is a Gröbner basis and $\langle G \rangle = I$. Since Gröbner bases are always taken with respect to some monomial order, we will let this monomial order be implicit. As such, if no monomial order is mentioned in a statement, the statement holds for any monomial order. Only if specific assumptions are required for the monomial order will it be mentioned.

We will also use the following alternative description of Gröbner bases.

\begin{definition}[Reduction modulo]
  Let $f, g \in A[X]$ be polynomials. We say $f$ \textit{reduces modulo} $g$ if $\LT(g) \mid \LT(f)$, since in that case $\LT(f - \gamma\cdot g) < \LT(f)$ where $\LT(f) = \gamma \cdot \LT(g)$ for some term $\gamma$. We say a polynomial reduces modulo a set of polynomials $G$ if there exists a $g \in G$ such that $f$ reduces modulo $g$. Finally, we say a polynomial \textit{reduces to zero} modulo $G$ if there is a chain of reductions that end in the zero polynomial.
\end{definition}

\begin{theorem}\label{thm:grb}
  Let $G \subset A[X]$. Then $\,G$ is a Gröbner basis if and only if every polynomial in $\langle G \rangle$ reduces to 0 modulo $G$.
\end{theorem}
\begin{proof}
  A good exercise.
\end{proof}

Furthermore, we have the following property of a Gröbner basis.

\begin{theorem}
  Let $G = \{g_{1}, \dots, g_{n}\} \subset k[X]$ be a Gröbner basis and let $f \in k[X]$. Then we can write
  \[f = r + \sum_{i=1}^{n} f_{i}g_{i}\]
  for some $f_{i} \in k[X]$ satisfying that $\LM(f_{i} g_{i}) \leq \LM(f)$ for each $i$. If $\,r \neq 0$, then $f \in \langle G \rangle$ if and only if the leading term of $r$ is not divisible by any leading term of $G$. Furthermore, if no term of $r$ is divisible by a leading term of $G$, then $r$ is unique.
\end{theorem}
\begin{proof}
  See chapter 2, proposition 1 and corollary 2 in~\cite{IVA}.
\end{proof}



A Gröbner basis need not be unique. Indeed, given a Gröbner basis G, we can add any element of $\langle G \rangle$ to $G$ and it is still a Gröbner basis. However, reduced Gröbner bases are unique.

\begin{definition}[Reduced Gröbner basis]
  A Gröbner basis $G$ is called \textit{reduced} if, for all $g \in G$, $g$ is a monic polynomial (i.e. $\LC(g) = 1$) and the only term of $g$ in $\langle \LT(\langle G \rangle) \rangle$ is $\LT(g)$.
\end{definition}
\begin{theorem}
  Let $I \subset k[X]$ be an ideal in a polynomial ring over a field. Then there is a unique reduced Gröbner basis of $I$.
\end{theorem}
\begin{proof}
  See chapter 2, theorem 5 of~\cite{IVA}.
\end{proof}

It is worth noting, that the second condition of reduced Gröbner bases is equivalent to every term of $g$ being irreducible modulo $G$, except for its leading term.
