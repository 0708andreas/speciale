\section*{Introduction}
Gröbner bases of ideals in multivariate polynomial rings are a vital tool when doing computational algebra and computational algebraic geometry. They often feature as an intermidiary step in all sorts of problems, ranging from intersecting ideals and solving polynomial equations, to backwards kinematics and determining the behaviour of chemical reaction networks\cite{IVA}. As such, the study of Gröbner bases have developed into a large field of its own.

In some situations, we may be interested in not just one ideal, but a family of ideals. For example, we might have a fixed algebraic curve, and we want to calculate the distance between a point and the curve. This can be done by computing a Gröbner basis for a system of equations, given by the Lagrange multipliers. Keeping the coordinates of the point as parameters, we would like to quickly compute this distance for many points. From a purely mathematical view, we might be interested in the behaviour of a Gröbner basis as we change the base ring. Given an ideal $I$ in a polynomial ring $K[U][X]$, we would like to describe the Gröbner basis obtained from $I$, by evaluating each $u \in U$ to a fixed value. In this way, we can see $I$ as a parameterized family of ideals, with $\,U$ being the parameters. A parametric Gröbner basis is a Gröbner basis of $\,I$, which remains a Gröbner basis for some choices of values for $U$ and a comprehensive parametric Gröbner basis remains a Gröbner basis for all choices of values for $U$. See definition~\ref{def:par_grb} for the precise definition.

Volker Weispfenning introduced the notion of a comprehensive parametric\footnote{Our terminology differs from the original terminology. What Weispfenning called a \textit{comprehensive Gröbner basis} we call a \textit{comprehensive parametric Gröbner basis.}} Gröbner basis in \cite{Weispfenning} in 1992 and gave an algorithm to compute them. He also gave some results on the computational complexity of this algorithm as well as some early applications. However, the computation of Gröbner bases, and by extension parametric Gröbner bases, is a difficult problem. Many optimizations and heuristics exist, which means the computation of Gröbner bases should be performed using highly optimized software. Reimplementing all this for parametric Gröbner bases might not be feasible. In 2006, Suzuki and Sato produced an algorithm for computing parametric Gröbner bases, that utilize existing software for computing Gröbner bases. Even though the theoretical complexity didn't change, the execution time certainly did, as the algorithm could exploit existing optimized algorithms for computing Gröbner bases. Kapur, Sun and Wang improved on this algorithm in 2010 \cite{KAPUR201327}\cite{10.1145/1837934.1837946}.

After parametric Gröbner bases were established, the search began for a unique object, a parametric analogue of the reduced Gröbner basis. Weispfenning introduced what he called a canonical comprehensive Gröbner basis in \cite{WEISPFENNING2003669}, which only depends on the ideal and the monomial order on the variables and the parameters. However, no object comparable to the reduced Gröbner basis, which was independent of assumptions on the ground ring was found until Michael Wibmer introduced Gröbner covers in \cite{grb_covers}. By drawing on machinery from modern algebraic geometry, including the language of sheaves and schemes, he found a parametric way of describing the reduced Gröbner basis of every specialization of the ideal in question, called Gröbner covers. He also proved the existence of a canonical Gröbner cover if $I$ is homogenous. This rather abstract paper was quickly followed up by \cite{MONTES20101391}, which described an algorithm to compute Gröbner covers. It should be noted, that even though the canonical Gröbner cover described by Wibmer is unique in a mathematical sense, there is no canonical, finite description of it, without adding in some assumptions on the base ring.

The focus of this paper is to serve as an introduction to parametric Gröbner bases. First, we establish parametric Gröbner bases, Gröbner systems and some initial results on parametric Gröbner bases. In particular, a fundamental theorem by Kalkbrener\cite{Kalkbrener} on when a Gröbner basis specializes to a Gröbner basis, as well as a tool called pseudo-division. Then, we cover the algorithm introduced by Suzuki and Sato. From here, we move on to Gröbner covers as introduced by Wibmer, to give an introduction into this, quite different framing of parametric Gröbner bases. We tie this theory together with the Suzuki-Sato algorithm, and provide plenty of examples to help get a feeling for the subject. Finally, we cover some applications of parametric Gröbner bases and Gröbner covers.

We purposefully do not cover Kapur, Sun and Wangs algorithm nor the implementation of Wibmers theory by Antonio Montes\cite{MONTES20101391}. These are seen as refinements of the material already covered, and covering them would exceed the scope of this introduction. Instead, we focus on tying the algorithm of Suzuki and Sato to the theory of Wibmer. This is how modern implementations of parametric Gröbner bases are implemented, but doesn't seem to be described in detail in the literature.

New contributions of this project include
\begin{itemize}
  \item Fixing edge-case bugs in the pseudo-code in \cite{ss_algo}, see algorithm~\ref{alg:CGS_simple}
  \item Modifying the Suzuki-Sato algorithm to produce Gröbner covers, see theorem~\ref{thm:CGS}
  \item Placing pseudo-division as a central tool of Gröbner covers and comprehensive Gröbner bases, see section~\ref{sec:ps_div} and \ref{sec:ps_div_app}
  \item Identifying a mistake in \cite{sturmfels} and fixing it using comprehensive Gröbner bases, see section~\ref{sec:bernd}
  \item A new implementation of comprehensive Gröbner bases in the Julia programming language with demonstrations of their applications.
\end{itemize}
