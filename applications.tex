\section{Applications}

\subsection{Quantifier elimination}
One of the first applications of parametric Gröbner bases was presented by its inventor Weispfenning \cite{Weispfenning} in the original article. It concerns the problem of computing a system of polynomial equations, whose solutions are equivalent to solutions to a set of logical expressions involving polynomial equations, con- and disjunctions, negations and existential quantifiers.

Sepcifically, we're given a formula $\exists x_{1}, \dots, x_{n} : \phi(U, x_{1}, \dots, x_{n})$ where $\phi$ is a combination using $\land$ and $\lor$ of polynomial equalities and inequalities in $k[U, X]$. If $k_{1}$ is an extension field of $k$, then that formula determines a partioning of $k_{1}^{|U|}$, namely those values of $U$ where the formula is true and those where it isn't. Our goal is to find a system of polynomial equations in $k[U]$ that is satisfied in exactly the same points.

First, we need to normalize the logical expressions, to fit a format we can work with.

\begin{definition}[Positive, primitive formula]
  A logical formula $\varphi$ is called \textit{positive and primitive} if it only involves polynomial equalities in $k[X]$, conjunctions and existential quantifiers.
\end{definition}

\begin{lemma}\label{lem:logical_positive}
  Let $\phi$ be a logical formula involving polynomial equalities, conjunctions, disjunctions, negations and existential quantifiers. Then there exists a finite set of positive, primitive formula $\varphi_{1}, \varphi_{2}, \dots, \varphi_{r}$ such that $\phi \iff (\varphi_{1} \lor \dots \lor \varphi_{r})$.
\end{lemma}
\begin{proof}
  Using standard logical rules, we can find $\phi_{1}, \dots, \phi_{r}$ containing only polynomial equalities, conjunction, negation and existential quantifiers such that \[\phi \iff \bigvee_{i=1}^{r} \phi_{r}.\] Using De Morgans law and distributivity we can assume that negations are at the lowest level of the formulas. Thus, we can see the $\phi_{i}$'s as existstential formulas containing conjunctions of polynomial equations and inequations.

  Now, to eliminate the inequalities, we use the following trick: \[f(X) \neq 0 \iff \exists\, t : f(X) \cdot t - 1 = 0.\]
\end{proof}

Thus we can solve each of the positive, primitive formulas independently, and see if any of them are satifiable.

\begin{theorem}
  Let $F \subset k[U, X]$ be a finite set of polynomials over an algebraically closed field and let $G$ be a parametric Gröbner basis of $F$. For a polynomial $f \in k[U][X]$, let $C(f) \subset k[U]$ denote the set of coefficients of non-constant terms in $f$. Then \[ \left(\exists x_{1}, \dots, x_{n} : \bigwedge_{f \in F} f(U, x_{1}, \dots, x_{n}) = 0 \right) \iff \bigwedge_{g \in G} \left( g(U, 0, \dots, 0) = 0 \lor \bigvee_{c \in C(g)} c(U) \neq 0 \right)\] in any extension field $k_{1} \supset k$.
\end{theorem}
\begin{proof}
  Let $\alpha \in k_{1}^{|U|}$. Then the question of whether $\exists x_{1}, \dots, x_{n} : \bigwedge_{f \in F} f(U, x_{1}, \dots, x_{n}) = 0$ is satisfied in $U = \alpha$ is equivalent to whether $\langle \sigma_{\alpha}(F) \rangle$ has a common zero, i.e. if $V(\langle \sigma_{\alpha}(F) \rangle) \neq \emptyset$.

  For the first implication, assume $\exists x_{1}, \dots, x_{n} : \bigwedge_{f \in F} f(U, x_{1}, \dots, x_{n}) = 0$ is satisfied at some $\alpha \in k_{1}^{|U|}$. Let $\beta \in k_{1}^{|X|}$ be a vector of $(x_{1}, \dots, x_{n})$ such that $f(\alpha, \beta) = 0$ for all $f \in F$. Then, since all $g \in G$ are also in $\langle F \rangle$, we get $g(\alpha, \beta) = 0 \; \forall g \in G$. Hence, if $g(\alpha, 0, \dots, 0) \neq 0$, then there has to be some non-constant term in $g$, which is also non-zero at $\alpha$.

  For the other implication, assume every $g \in G$ has zero constant term or some non-zero non-constant term, when viewed as a polynomial in $k[U][X]$. Assume for a contradiction that $V(\langle \sigma_{\alpha}(F) \rangle) = \emptyset$. By the weak Nullstellensatz we get that $1 \in \langle \sigma_{\alpha}(F) \rangle$. Since $G$ is a parametric Gröbner basis, there is some $g \in G$ such that $\LT(\sigma_{\alpha}(g)) \mid 1$. Thus $\sigma_{\alpha}(g)$ is a constant polynomial with non-zero constant term, contradicting the assumption.
\end{proof}










\subsection[Bernds conjecture]{Bernds conjecture\footnote{Named such by Bernd Sturmfels in a private communication to the supervisor of this project.}}
In the article \cite{sturmfels}, Bernd Sturmfels states the following theorem without proof.

\begin{theorem}
  Let $K$ be an algebraically closed field and $F = \{f_{1}, \dots, f_{k}\} \subset K[x_{1}, \dots, x_{n}]$ a finite set of polynomials. Assume that $\V(F) = \emptyset$ and consider the ideal $\langle y_{1} - f_{1}, \dots, y_{k} - f_{k} \rangle \subset K[x_{1}, \dots, x_{n}, y_{1}, \dots, y_{k}]$. Let $G$ be a Gröbner basis of $I$ with respect to the lexicographic order with $x_{1} > \dots > x_{n} > y_{1} > \dots > y_{k}$. Then $G$ contains a polynomial $p$ (called a final polynomial) such that
  \begin{enumerate}
    \item $p(x_{1}, \dots, x_{n}, 0, 0, \dots, 0) \in K$
    \item $p(x_{1}, \dots, x_{n}, f_{1}, \dots, f_{k}) = 0$.
  \end{enumerate}
\end{theorem}

He writes that the proof is ``straightforward but fairly technical''. Here is a relatively clean proof, using the theory of parametric Gröbner bases and pseudo-division.

\indent \begin{proof}
  We will only prove that such a Gröbner basis contains a polynomial with the first property, and will refer to such a polynomial as a final polynomial. Let $I = \langle y_{1} - f_{1}, \dots, y_{k} - f_{k} \rangle \subset K[y_{1}, \dots, y_{k}][x_{1}, \dots, x_{n}]$, and let $G = \{g_{1}, \dots, g_{t}\} \subset K[x_{1}, \dots, x_{n}, y_{1}, \dots, y_{k}]$ be a Gröbner basis of $I \subset K[x_{1}, \dots, x_{n}, y_{1}, \dots, y_{k}]$ w.r.t.\ the lexicographic order with $x_{1} > \dots > x_{n} > y_{1} > \dots > y_{k}$. Since every product of $y$'s is smaller than any product of $x$'s, $G$ can be seen as a Gröbner basis of $I \subset K[y_{1}, \dots, y_{k}][x_{1}, \dots, x_{n}]$ (source: trust me, bro).

  First, $I$ must contain a final polynomial. Indeed, let $\mathcal G$ be a parametric Gröbner basis of $I$ and let $\sigma : K[y_{1}, \dots, y_{k}] \to K$ be the specialization setting every $y_{i}$ to 0. Since $\langle \sigma(I) \rangle = \langle F \rangle$, and $\langle F \rangle = \langle 1 \rangle$ be the Nullstellensatz, there must be some $g \in \mathcal G$ such that $\LM(\sigma(g)) \mid 1$, hence $g$ is a final polynomial.

  Now let $p \in I$ be a final polynomial, and by rescaling we can assume $\sigma(p) = 1$. Since $G$ is a Gröbner basis, we can apply the normal division algorithm to write
  \[p = \sum_{i=1}^{t} g_{i} h_{i}\]
  where $\LM(g_{i}h_{i}) \leq \LM(p)$ and $\coef(h_{i}, m) \in \langle \coef(p, m') \mid m' \geq \LM(g_{i}m) \rangle$ for all monomials $m$. Since this is in particular a pseudo-division, we get that
  \[1 = \sigma(p) = \sum_{i=1}^{t} \sigma(g_{i} h_{i})\]
  and $\LM(\sigma(g_{i} h_{i})) \leq \LM(\sigma(p)) = 1$ by lemma~\ref{lem:ps_div_to_div}. Hence, every $g_{i}$ where $h_{i} \neq 0$ satisfies $\LM(\sigma(g_{i})) = 1$ implying $\sigma(g_{i}) \in K \setminus \{0\}$. This means $g_{i}$ is a final polynomial.
\end{proof}


% \indent \begin{proof}
%   We will only prove the existence of a polynomial satisfying the first property, and will refer to any polynomial satisfying that property as a final polynomial. Let $I = \langle y_{1} - f_{1}, \dots, y_{k} - f_{k} \rangle \subset K[y_{1}, \dots, y_{k}][x_{1}, \dots, x_{n}]$ and let $\mathcal G$ be a parametric Gröbner basis of $I$. Let $\sigma : K[y_{1}, \dots, y_{k}] \to K$ be the specialization setting every $y_{i}$ to 0. Then $\mathcal G$ contains a final polynomial. To see this, notice that $\sigma(\mathcal G)$ is a Gröbner basis of $\langle \sigma(I) \rangle$ which, by the Nullstellensatz, is equal to $\langle 1 \rangle$. Hence there is some $g \in \mathcal G$ such that $\LM(\sigma(g)) \mid 1$, implying that $\sigma(g) \in K \setminus \{0\}$.

%   Now, we let $\mathcal G' = \{g \in \mathcal G \mid \sigma(g) \notin K\} = \{g_{1}, \dots, g_{t}\}$. Assume that $\mathcal G'$ is still a Gröbner basis for $I$ w.r.t. the lexicographic order. Since $I$ contains a final polynomial, say $p$, we can apply the division algorithm to write
%   \[p = \sum_{i=1}^{t} g_{i}h_{i}\]
%   which in particular is a pseudo-division. Since pseudo-divisions are stable under specializations by lemma~\ref{lem:ps_div_to_div}, we get that
%   \[1 = \sigma(p) = \sum_{i=1}^{t} \sigma(g_{i} h_{i})\]
%   and $\LM(\sigma(g_{i}h_{i})) \leq 1$. Hence, every $g_{i}$ where $h_{i} \neq 0$ satisfies $\sigma(g) \in K$, but that is a contradiction. Thus $\mathcal G'$ is not a Gröbner basis of $I$.

%   To drive it home, let $G$ be a Gröbner basis of $I \subset K[x_{1}, \dots, x_{n}, y_{1}, \dots, y_{k}]$ w.r.t. the lexicographic order and assume $G$ doesn't contain a final polynomial. Since any product of $y$'s is smaller than any product of $x$'s, we can see $G$ as a Gröbner basis of $I \subset K[y_{1}, \dots, y_{k}][x_{1}, \dots, x_{n}]$. Then, there is a parametric Gröbner basis $\mathcal G$ containing $G$. Let $\mathcal G' = \{g \in \mathcal G \mid \sigma(g) \notin K\}$. Then $G \subset \mathcal G'$, but $\mathcal G'$ is not a Gröbner basis by the above. Thus $G$ cannot be a Gröbner basis.
% \end{proof}















% First, we need a lemma.

% \begin{lemma}\label{lem:LT_is_uncontaminated}
%   Let $\{f_{1}, \dots, f_{k}\} \subset K[x_{1}, \dots, x_{n}]$ and let $G$ be the reduced Gröbner basis of the ideal $\langle y_{1} - f_{1}, \dots, y_{k} - f_{k} \rangle \subset K[x_{1}, \dots, x_{n}, y_{1}, \dots, y_{k}]$ w.r.t. the lexicographic order with $x_{1} > \dots > x_{n} > y_{1} > \dots > y_{k}$. Then for all $g \in G$, either $g \in K[y_{1}, \dots, y_{k}]$ or $\LT(g) \in K[x_{1}, \dots, x_{n}]$.
% \end{lemma}
% \begin{proof}
%   Note that the given generators of $I$ have the property, that if we write the terms of any generator in order (by the term order), then there is a term, only containing $y$'s, such that every term before it only contains $x$'s and every term after it only contains $y$'s. We wish to keep this invariant.

%   We use Buchbergers algorithm to compute a Gröbner basis. Assume that at the beginning of a certain step in the algorithm, the above invariant is satisfied. Suppose we want to reduce some S-polynomial
%   \[S(f_{i}, f_{j}) = \frac{\lcm(\LM(f_{i}), \LM(f_{j}))}{\LT(f_{i})} f_{i} - \frac{\lcm(\LM(f_{i}), \LM(f_{j}))}{\LT(f_{j})} f_{j}\]
%   If both of
% \end{proof}
